\documentclass[a4paper,12pt]{article}
\usepackage[portuguese]{babel}
\usepackage{amsmath,amsthm,amssymb,mathtools,tikz,graphicx,pgfplots,verbatim,fancyhdr}
\pagestyle{fancy}%Style of page
\setlength{\headsep}{0.5in}

\lhead{%Header left
	Universidade Federal da Para\'iba \\
	Discente: Jefferson Bezerra dos Santos\\
	Docente: Thiago Jos\'e Machado
}

\title{ Terceira lista de EDO te\'orica}
\date{\vspace{-5ex}}

\begin{document}
\maketitle\thispagestyle{fancy}
\textbf{1. Quest\~ao:}\\
a) $y^{''} + 4y$ = 0\\
Equa\c c\~ao homog\^enia auxiliar:
\begin{align*}
	{\lambda}^{2} + 4 = 0 \\
	{\lambda}^{2} = \pm 2i
\end{align*}
A solu\c c\~ao homog\^enea \'e da forma:
\begin{align*}
	y_{h}{x} &= c_1e^{-2i} + c_2 e^{2i}\\
	y_{h}{x} &=  c_1 \left [ cos(2x) -isen(2x)\right ] +  c_2 \left [ cos(2x) +isen(2x)\right ]\\
	y_h{x} &= (c_1 + c_2)cos(2x) + (-c_1 + c_2)isen(2x)
\end{align*}
tomando $b_1 = c_1 + c_2$ e $b_2 = (-c_1 + c_2)i$ temos que
\begin{align*}
	y_{h} &= b_1 cos(2x) + b_2 sen(2x)
\end{align*}
segue que $y_1 = cos(2x)$ e $y_2 = sen(2x)$. Portanto, o conjunto ser\'a dada por,
$$C =\left \{ cos(2x), sen(2x) \right \}$$
para a verifica\c c\~ao da indep\^encia vamos utilizar o Wronskiano, 
\begin{align*}
	W(y_1,y_2) = det \left [ \begin{array}{cc} 
			y_1 & {y_1}^{'} \\
			y_2 & {y_2}^{'}
	\end{array} \right ] 
\end{align*}
\begin{align*}
	W(y_1,y_2) = det \left [ \begin{array}{cc} 
			cos(2x) & -2sen(2x) \\
			sen(2x) & 2cos(2x)
	\end{array} \right ] = 2({sen}^{2}(x) + {cos}^{2}(x)) = 2.1 \neq 0.
\end{align*}
Logo, o conjunto C \'e independente. \\

b)$y^{(3)} -6y^{(2)} + 11y^{(1)} - 6y = 0$\\

Equa\c c\~ao homog\^enea: \\
${\lambda}^{3} -6{\lambda}^{2} + 11{\lambda} - 6 = 0$\\
note que $\lambda = 1$ \'e uma das ra\'izes da equa\c c\~ao homog\^enea, portanto aplicando a divis\~ao de polin\^omios
temos que \\
${\lambda}^{3} -6{\lambda}^{2} + 11{\lambda} - 6 = (\lambda - 1)({\lambda}^{2} -5{\lambda} + 6) =  (\lambda - 1)(\lambda - 2)(\lambda - 3)$
\\ logo,\\
$$y_h(x) = c_1e^{x} + c_2e^{2x} + c_3e^{3x}$$
o conjunto C a ser verificado \'e da forma
$$C = \{e^{x}, e^{2x}, e^{3x}\}.$$
O wronskiano \'e dado por,
\begin{align*}
	W(y_1,y_2, y_3) = det \left [ \begin{array}{ccc} 
			e^{x} & e^{2x}  & e^{3x}\\
			e^{x} & 2e^{2x} & 3e^{3x}\\
			e^{x} & 4e^{2x} & 9e^{3x}
	\end{array} \right ] = e^{x}(8e^{5x} - 6e^{5x})= e^{x}(2e^{5x}) \neq 0 \quad \forall x \in \mathbb{R}.
\end{align*}
c) $y^{4} - y = 0$ \\
Equa\c c\~ao homog\^enea :\\
${\lambda}^{4} - 1 = 0$\\
cujas as ra\'izes s\~ao\\
${\lambda}_1 = -1,  {\lambda}_2 = 1, {\lambda}_3 = -i \hspace{3pt} e\hspace{3pt} {\lambda}_4 = i.$

$$y_h(x) = c_1e^{-x} + c_2e^{x} + c_3e^{-ix} + c_4e^{ix}$$
$$y_h(x) = c_1e^{-x} + c_2e^{x} + b_1cos(x) + b_2sen(x)$$
O conjunto \'e dado por,
$$C = \{e^{-x},e^{x}, cos(x),sen(x)\}$$
\begin{align*}
	W(y_1,y_2, y_3, y_4) = det \left [ \begin{array}{cccc} 
			e^{-x}  & e^{x} & cos(x)  &  sen(x)\\
		   -e^{-x}  & e^{x} & -sen(x) &  cos(x)\\
			e^{-x}  & e^{x} & -cos(x) & -sen(x)\\
		   -e^{-x}  & e^{x} & sen(x)  & -cos(x)
	\end{array} \right ] = e^{-x}(2e^{x})\left | 
		\begin{array}{cc}	
			-2cos(x) & -2sen(x) \\
			2sen(x) & -2cos(x)
		\end{array}
	\right |
	\\= 4 \neq 0.
\end{align*}
portanto, o conjunto C \'e L.I.\\

d) $y^{4} -6y^{3} + 18y^{2} -30y^{1} + 25y= 0$ \\ 
Equa\c c\~ao homog\^nea:\\
${\lambda}^{4} -6{\lambda}^{3} + 18{\lambda}^{2} -30{\lambda} + 25= 0$\\
${\lambda}_1 = 1-2i,  {\lambda}_2 = 1+2i, {\lambda}_3 = 2-i \hspace{3pt} e\hspace{3pt} {\lambda}_4 = 2+i.$ \\
A solu\c c\~ao homog\^enea \'e da forma,
$$y_h(x) = c_1e^{(1-2i)x} + c_2e^{(1+2i)x} + c_3e^{(2-i)x} + c_4e^{(2+i)x}$$
logo, o conjunto C \'e da forma 
$$C = \{e^{(1-2i)x},e^{(1+2i)x},e^{(2-i)x},e^{(2+i)x}\}$$
$$W = -160e^{6x} \neq 0$$
logo, o conjunto C \'e L.I. Para os c\'alculos foi utilizado o Wolfram.\\
e) $y^{3} -3y  = 0$\\
Equa\c c\~ao homog\^enea: \\
${\lambda}^{3} -3{\lambda} = 0$\\
${\lambda}({\lambda}^{2} -3) = 0$\\
${\lambda}_1 = 0,{\lambda}_2 = -\sqrt{3},{\lambda}_3 = \sqrt{3}$\\
o conjunto C a ser verificado \'e da forma
$$C = \{1, e^{(-\sqrt{3})x},e^{(\sqrt{3})x}\}.$$
\begin{align*}
	W(y_1,y_2, y_3,) = det \left [ \begin{array}{ccc} 
			1 & 0  & 0\\
			e^{(-\sqrt{3})x} & -\sqrt{3}e^{(-\sqrt{3})x} & 3e^{(-\sqrt{3})x}\\
e^{(\sqrt{3})x} & \sqrt{3}e^{(\sqrt{3})x} & 3e^{(\sqrt{3})x}\\
	\end{array} \right ]= -6\sqrt{3} \neq 0.
\end{align*}
portanto, o conjunto C \'e L.I.\\
f)$y^{5} - y^{4} -2y^{3} + 2y^{2} + y^{1} -y = 0$\\
Equa\c c\~ao homog\^enea:\\
${\lambda}^{5} - {\lambda}^{4} -2{\lambda}^{3} + 2{\lambda}^{2} + {\lambda}^{1} -1$\\
uma das ra\'izes \'e $\lambda -1 = 0$, portanto aplicando  aplicando divis\~ao de polin\^omio obtemos que
\begin{align*}
({\lambda}^{4} -2{\lambda}^{2} + 1) (\lambda -1) = 0\\
({\lambda}^{2} -1)^{2} (\lambda -1) = 0\\
({\lambda} + 1)^{2} (\lambda -1)^{3} = 0
\end{align*}
logo, a solu\c c\~ao homog\^enea \'e da forma
\begin{align*}
	y_h(x) = c_1e^{-x} + c_2xe^{-x} + c_3e^{x} + c_4xe^{x} + c_4{x}^{2}e^{x}
\end{align*}
o conjunto C a ser verificado \'e da forma \\
$$C = \{e^{-x}, xe^{-x},e^{x},xe^{x},{x}^{2}e^{x}\}.$$
segue que 
\begin{align*}
	W = 128 e^{x} \neq  0
\end{align*}
portanto o conjunto C \'e L.I para o c\'alculo de W foi utilizado Wolfram.

\textbf{2. Quest\~ao:}
\begin{align*}
	y^{(4)}(x) + 2y^{(2)}(x) + y(x) = 3e^{2x} + 2sin(x) - 8e^{x}cos(x)
\end{align*}
Equa\c c\~ao homog\^enea:\\
\begin{align*}
{\lambda}^{4} + 2 {\lambda}^{2} + 1 = 0\\
({\lambda}^{2} + 1)^{2} = 0\\
({\lambda}^{2} + 1) = 0\\
{\lambda} = \pm i
\end{align*}
solu\c c\~ao homog\^enea \'e da forma,
\begin{align*}
	y_h(x) &= c_1e^{ix} + c_2e^{-ix} + c_3xe^{ix} + c_4xe^{-ix} \\
	&=  k_1cos(x) + k_2sen(x) + k_3xcos(x) + k_4xsen(x)
\end{align*}
a solu\c c\~ao particular \'e da forma, 
\begin{align*}
	y_p(x) = ae^{2x} + bcos(x) + csen(x) + de^{x}cos(x) + {\alpha} e^{x} sen(x)
\end{align*}
calculando a derivadas da solu\c c\~ao particular segue que
\begin{align*}
	{y_p(x)}^{(1)} &= 2ae^{2x} -b sen(x) + cos(x) de^{x}(cos(x) -sen(x) + {\alpha}e^{x} (sen(x) + cos(x))) \\
	{y_p(x)}^{(2)} &= 4ae^{2x} -bcos(x) -csen(x) + de^{x}(-2sen(x)+ {\alpha}e^{x}(2cos(x)))\\
	{y_p(x)}^{(3)} &= 8ae^{2x} + bsen(x) -ccos(x) + de^{x}(-2cos(x) -2sen(x)) + {\alpha}e^{x}(2cos(x) -2sen(x)) \\
	{y_p(x)}^{(4)} &= 16ae^{2x} + bcos(x) + csen(x) + de^{x}(-4cos(x)+ {\alpha}e^{x}(-4sen(x)))
\end{align*}
substituindo na equa\c c\~ao segue que 
\begin{align*}
	y^{(4)}(x) + 2y^{(2)}(x) + y(x) &= \\ 
	&= 25ae^{2x} + (-4de^{x} -3{\alpha}e^{x})sen(x) + (-3d + 4{\alpha})e^{x}cos(x)\\
	&= 3e^{2x} + 2sin(x) - 8e^{x}cos(x)
\end{align*}
da igualdade acima obtemos o seguinte sistema 
\begin{align*}
	\left \{ 
{\arraycolsep=3.4pt\def\arraystretch{1.5}
		\begin{array}{cl}	
			25a &= 3 \\
			-4d -3{\alpha} &= \frac{2}{e^{x}} \\
			-3d -4{\alpha} &= -8	
		\end{array}
	}
		\right . \Rightarrow  
		\left \{ 
{\arraycolsep=3.4pt\def\arraystretch{1.5}
		\begin{array}{cl}	
			a &= \frac{3}{25} \\
			{\alpha} &= -\frac{1}{25}\left ( 32 + \frac{6}{e^{x}}\right )  \\
			d &= \frac{4}{3}\left [ -\frac{1}{25}(32 + \frac{6}{e^{x}})\right ] 	
		\end{array}
	}
		\right .
\end{align*}
portanto, a solu\c c\~ao geral \'e da forma
\begin{align*}
	y(x) = y_h(x) + y_p(x)
\end{align*}
onde :
\begin{align*}
a &= \frac{3}{25} \\
			{\alpha} &= -\frac{1}{25}\left ( 32 + \frac{6}{e^{x}}\right )  \\
			d &= \frac{4}{3}\left [ -\frac{1}{25}(32 + \frac{6}{e^{x}})\right ] 	
\end{align*}
com $b,c \in \mathbb{R}$.

\textbf{3. Quest\~ao:}\\
Seja $y_1$ e $y_2$ as fun\c c\~oes solu\c c\~ao. Temos que para que sejam independentes \'e suficiente que 
\begin{align*}	
	W(y_1,y_2) \neq 0 \Rightarrow x^{2} - 4 \neq  0 \Rightarrow x \neq \pm 2
\end{align*}
contudo, nenhuma restri\c c\~ao do dom\'inio de $x^{2} - 1$ foi especificada, portanto, as fun\c c\~oes $y_1$ e $y_2$
s\~ao L.D.

\textbf{4. Quest\~ao:}\\
Seja a equa\c c\~ao 
\begin{align*}
	ay^{(2)}(x) + by^{(1)}(x) + cy(x) = f(x)
\end{align*}
da equa\c c\~ao homog\^enea obtemos que 
\begin{align*}
	{\lambda} = \frac{-b \pm \sqrt{b^{2} - 4ac}}{2a}
\end{align*}
vejamos primeiramente para o caso real, tomando 
\begin{align*}
	\alpha =  \frac{-b + \sqrt{b^{2} - 4ac}}{2a}\\
	\beta =  \frac{-b + \sqrt{b^{2} - 4ac}}{2a}\\
\end{align*}
e supondo que $\alpha < 0$ e $\beta < 0$ temos que 
\begin{align*}
	y_1(x) = e^{{\alpha}x} \Rightarrow \lim_{x \rightarrow \infty}  e^{{\alpha}x} = 0\\
	y_2(x) = e^{{\beta}x} \Rightarrow \lim_{x \rightarrow \infty}  e^{{\beta}x} = 0\\
\end{align*}
segue que 
\begin{align*}
	y_1(x) - y_2 (x) = 0  \quad \textrm{quando} \quad x \rightarrow \infty
\end{align*}
supondo que  $\alpha < 0$ e $\beta < 0$ onde $\alpha = \beta$ segue que 
\begin{align*}
	y_1(x) &= e^{{\alpha}x} \Rightarrow \lim_{x \rightarrow \infty}  e^{{\alpha}x} = 0\\
	y_2(x) &= xe^{{\beta}x} 
\end{align*}
reescrevendo e aplicando a regra de L'Hopital em $y_2$ temos que
\begin{align*}
	\lim_{x \rightarrow \infty}  xe^{{\beta}x} = \lim_{x \rightarrow \infty}  \frac{x}{\frac{1}{e^{{\beta}x}}} =
	\lim_{x \rightarrow \infty} \frac{1}{e^{{-\beta}x}} = 0\\
\end{align*}
logo, 
\begin{align*}
	y_1(x) - y_2 (x) = 0  \quad \textrm{quando} \quad x \rightarrow \infty
\end{align*}
note que o resultado \'e v\'alido para o caso que $\alpha < 0$ e $\beta < 0 $ . Vejamos para o caso que $\lambda$ \'e
complexo, portanto $\lambda$ \'e da forma 
\begin{align*}
	{\lambda} = \frac{-b \pm ki}{2a}
	\end{align*}
onde $ki = \sqrt{b^{2} - 4ac}$.
Portanto, temos que para $\alpha = \beta,$
\begin{align*}
	y_1 &= e^{(\frac{-b - ki}{2a})x} =  {e^{\frac{-b}{2a}x}}(cos(\frac{-kx}{2a}) + isen(\frac{-kx}{2a})) \\
	y_2 &= e^{(\frac{-b + ki}{2a})x} =  {e^{\frac{-b}{2a}x}}(cos(\frac{kx}{2a}) + isen(\frac{kx}{2a}))
\end{align*}
como as fun\c c\~oes $cos, sen$ s\~ao limitadas segue que 
\begin{align*}
	\lim_{x \rightarrow \infty} y_1 = 0 \\
	\lim_{x \rightarrow \infty} y_2 = 0 \\
\end{align*}
logo, 
\begin{align*}
	y_1(x) - y_2 (x) = 0  \quad \textrm{quando} \quad x \rightarrow \infty
\end{align*}
para o caso de $\alpha = \beta$ teriamos que 
\begin{align*}
	y_1 &= e^{(\frac{-b - ki}{2a})x} =  {e^{\frac{-b}{2a}x}}(cos(\frac{-kx}{2a}) + isen(\frac{-kx}{2a})) \\
	y_2 &= xe^{(\frac{-b + ki}{2a})x} =  x{e^{\frac{-b}{2a}x}}(cos(\frac{kx}{2a}) + isen(\frac{kx}{2a}))
\end{align*}
como no caso anterior as fun\c c\~oes $sen, cos$ s\~ao limitadas e o termo $xe^{\frac{-bx}{2a}} \rightarrow 0$ quando $x
\rightarrow \infty$. Portanto,
\begin{align*}
	y_1(x) - y_2 (x) = 0  \quad \textrm{quando} \quad x \rightarrow \infty.
\end{align*}
note que a prova \'e v\'alida desde que para o caso real tenhamos $\alpha < 0$ e $\beta < 0$.

Para $b = 0$ o resultado cont\'inua v\'alido, pois analisando o termo 
\begin{align*}
	\sqrt{b^{2} - 4ac}
\end{align*}
caso esse seja 0 teriamos 
\begin{align*}
	y_1 = 1 \\
	y_2 = 1 
\end{align*}
e portanto,
\begin{align*}
	y_1(x) - y_2 (x) = 0  \quad \textrm{quando} \quad x \rightarrow \infty.
\end{align*}
as outras op\c c\~oes seriam admitir um valor real ou imagin\'ario que iriam recair nos casos j\'a demonstrados. Note
que para o caso as provas feitas s\~ao verdadeiras desde que $\alpha < 0$  e $\beta < 0$.

b) Dada a equa\c c\~ao 
\begin{align*}
	ay^{(2)}(x) + by^{(1)}(x) + cy(x) = f(x)
\end{align*}
onde 
\begin{align*}
	f(x) = d
\end{align*}
tomemos uma solu\c c\~ao particular da forma
\begin{align*}
	y_p(x) = \frac{d}{c}
\end{align*}
temos que a solu\c c\~ao geral \'e da forma 
\begin{align*}
	y(x) = y_h(x) + y_p(x)
\end{align*}
pela letra (a) temos que
\begin{align*}
	\lim_{x \rightarrow \infty} y_h(x)  = 0
\end{align*}
nota-se que 
\begin{align*}
	\lim_{x \rightarrow \infty} y_p(x) = \frac{d}{c}
\end{align*}
portanto,
\begin{align*}
	y(x) \rightarrow \frac{d}{c} \quad \textrm{quando } \quad x \rightarrow  \infty
\end{align*}

\textbf{5. Quest\~ao:}\\

a)$x^{2}y^{(2)}(x) + 7xy^{1}(x) + 4y(x) = ln(x^{-3})$\\
Vamos as seguintes substitui\c c\~aoes :
\begin{align*}
	x &= e^{t} \\
	y^{(1)} &= \frac{1}{x} (\dot y) \\
	y^{(2)} &= \frac{1}{x^{2}} (\ddot y - \dot y)
\end{align*}
seque que 
$\ddot y + 6 \dot y + 4y= -3t$ \\
Equa\c c\~ao homog\^enea:\\
\begin{align*}
{\lambda}^{2} +6{\lambda} + 4 = 0\\
{\lambda} = -3 \pm \sqrt{5}
\end{align*}
solu\c c\~ao homog\^enea \'e da forma 
\begin{align*}
	y_h(t) = c_1 e^{(-3 -\sqrt{5})t} + c_2 e^{(-3 +\sqrt{5})t} 
\end{align*}
a particular ser\'a dada por
\begin{align*}
	y_p(t) = {\beta}_1 t + {\beta}_0 
\end{align*}
de onde temos que 
\begin{align*}
	{y_p}^{(1)}(t) &= {\beta}_1 \\
	{y_p}^{(2)}(t) &= 0 
\end{align*}
segue que
\begin{align*}
	6{\beta}_1 + 4({\beta}_1 t+ {\beta}_0) = -3t 
\end{align*}
\[\left \{ 
	\begin{array}{cll}
		4{\beta}_1 &= -3 \Rightarrow {\beta}_1 &=- \frac{3}{4}\\
		6{\beta}_1 + 4{\beta}_0 &= 0 \Rightarrow {\beta}_0 &=\frac{9}{8} 
	\end{array}
\right .
\]
logo, 
\begin{align*}
	y_p(t) = - \frac{3}{4}t + \frac{9}{8}
\end{align*}
a solu\c c\~ao geral \'e 
\begin{align*}
	y(t) &= y_h(t) + y_p(t) \\
	&=  c_1 e^{(-3 -\sqrt{5})t} + c_2 e^{(-3 +\sqrt{5})t} - \frac{3}{4}t + \frac{9}{8}
\end{align*}
como $x = e^{t}$ temos que a solu\c c\~ao geral \'e da forma 
\begin{align*}
	y(x) = c_1 x^{(-3 -\sqrt{5})} + c_2 x^{(-3 +\sqrt{5})}  - \frac{3}{4} ln(x) + \frac{9}{8}
\end{align*}

b)$x^{3}y^{(3)}(x) - 3x^{2}y^{(2)}(x) + 6xy^{(1)}(x) = x$\\
Vamos as seguintes substitui\c c\~oes :
\begin{align*}
	x &= e^{t} \\
	y^{(1)} &= \frac{1}{x} (\dot y) \\
	y^{(2)} &= \frac{1}{x^{2}} (\ddot y - \dot y) \\
	y^{(3)} &= \frac{1}{x^{3}} ( \dddot y -3 \ddot y +2 \dot y)
\end{align*}
segue que 
\begin{align*}
	\dddot y - 6 \ddot y + 11\dot y - 6 y = e^{t}
\end{align*}
equa\c c\~ao homog\^enea \'e da forma
\begin{align*}
	{\lambda}^{3} -6{\lambda}^{2} + 11{\lambda}^{1} - 6 = 0\\
	{(\lambda - 1)}{(\lambda - 2)} {(\lambda - 3)} = 0
\end{align*}
a solu\c c\~ao homog\^enea \'e da forma 
\begin{align*}
	y_h (t) = c_1 e^{t} + c_2 e^{2t} + c_3 e^{3t}
\end{align*}
a solu\c c\~ao particualar \'e da forma 
\begin{align*}
	y_p(t) = {\alpha}t e^{t}
\end{align*}
seque que 
\begin{align*}
	\dot {y_p} (t) &= {\alpha}(t e^{t} + e^{t}) \\
	\ddot {y_p} (t) &= {\alpha}(t e^{t} + 2e^{t}) \\
	\dddot {y_p} (t) &= {\alpha}(t e^{t} + 3e^{t}) 
\end{align*}
substituindo temos que 
\begin{align*}
	{\alpha}(t e^{t} + 3e^{t}) - 6({\alpha}(t e^{t} + 2e^{t})) + 11({\alpha}(t e^{t} + e^{t}))  - 6 ({\alpha}t e^{t}
	) = e^{t} 
\end{align*}
logo,
\begin{align*}
	{\alpha} = -\frac{1}{4}
\end{align*}
portanto,
\begin{align*}
	y(t) &= c_1 e^{t} + c_2 e^{2t} + c_3 e^{3t} -\frac{1}{4}
 t e^{t} \\
 y(x) &= c_1 x + c_2 x^{2} + c_3 x^{3} -\frac{1}{4}xln(x)
\end{align*}

c) $4x^{2}y^{(2)} -5xy^{(1)} + y = x^{2}$ \\
Vamos utilizar o m\'etodo de Frobenius\\
Equa\c c\~ao indicial:
\begin{align*}
	A{\lambda}^{2} + (-A + B){\lambda} + c = 0  \\
	4{\lambda}^{2} -  9{\lambda} + 1 = 0  \\
	{\lambda} = \frac{9 \pm \sqrt{65}}{8}
\end{align*}
solu\c c\~ao homog\^enea \'e da forma 
\begin{align*}
	y_h(x) = c_1 e^{(\frac{9 - \sqrt{65}}{8})x} + c_2 e^{(\frac{9 + \sqrt{65}}{8})x} 
\end{align*}
a solu\c c\~ao particualar \'e da forma 
\begin{align*}
	y_p(x) = {\beta}_2 x^{2} + {\beta}_1 x + {\beta}_0
\end{align*}
segue que 
\begin{align*}
	{y_p}^{(1)} &= 2{\beta}_2 x + {\beta}_1  \\
	{y_p}^{(2)} &= 2{\beta}_2  \\
\end{align*}
portanto, temos que 
\begin{align*}
	4x^{2} (2{\beta}_2) -5x(2{\beta}_2 x + {\beta}_1) + {\beta}_2 x^{2} + {\beta}_1 x + {\beta}_0 = x^{2}
\end{align*}
da equa\c c\~ao acima temos que 
\[ \left \{ \begin{array}{cc}
		{\beta}_2  &= -1 \\
		{\beta}_1  &= 0\\
		{\beta}_0  &= 0 \\
	\end{array} \right . 
\]
segue que
\begin{align*}
	y_p(x) = -x^{2}
\end{align*}
a solu\c c\~ao geral \'e da forma 
\begin{align*}
	y(x) = y_h(x) + y_p(x)
\end{align*}

\textbf{6. Quest\~ao:}\\

a)
$$ x^{2}y^{(2)} -2y(x) = 3x^{2} - 1, x > 0$$ 
$$y_1(x) = x^{2}, y_2(x) = x^{-1}$$
vamos verificar que $y_1$ e $y_2$ s\~ao solu\c c\~ao da homog\^enea temos que 
\begin{align*}
	y_h &= c_1 x^{2}  + c_2 x^{-1} \\
	{y_h}^{(1)} &= 2 c_1x -c_2 x^{-2} \\
	{y_h}^{(2)} &= 2 c_1 +2c_2 x^{-3}
\end{align*}
substituindo temos que 
\begin{align*}
	x^{2} (2 c_1 +2c_2 x^{-3}) - 2(c_1 x^{2}  + c_2 x^{-1}) = 0 + 0 = 0
\end{align*}
solu\c c\~ao verificada. Calculando o wronskiano
\[ W = \left [ 
	\begin{array}{cc}
		x^{2} & 2x \\
		x^{-1} & -x^{-2}
	\end{array}
\right ] = -3 \neq 0 \Rightarrow \mbox{L.I}
\]
a solu\c c\~ao particular \'e da forma 
\begin{align*}
	y_p(x) = u_1 y_1 + u_2 y_2 
\end{align*}
vamos calcular $u_1$ e $u_2$
\begin{align*}
	u_1 = - \int \frac{y_2 f}{a_2 W} dx &= \frac{1}{3} \int \frac{(3x^{2} -1)}{x^{3}} dx \\ 
	& = \frac{1}{3}\left [ 3ln(x) + \frac{1}{2}x^{-2} + k_1\right ] 
\end{align*}

\begin{align*}
	u_2 =  \int \frac{y_1 f}{a_2 W} dx &= -\frac{1}{3} \int (3x^{2} -1) dx\\
	&= -\frac{1}{3}\left [ x^{3} - x + k_2\right ] 
\end{align*}

b)
\begin{align*}
	x^{2}y^{(2)} -x(x + 2)y^{(1)} + (x + 2)y = 2x^{3}, x > 0 \\
	y_1 = x, y_2 = xe^{x}
\end{align*}
vamos verificar se $y_1$ e $y_2$ s\~ao solu\c c\~ao da homog\^enea, temos que 
\begin{align*}
	y_h &= c_1x + c_2 xe^{x}\\ 
	{y_h}^{(1)} &= c_1 + c_2(xe^{x} + e^{x}) \\ 
	{y_h}^{(2)} &= c_2(xe^{x} + 2e^{x}) \\ 
\end{align*}
substituindo temos que 
\begin{align*}
	x^{2}(c_2(xe^{x} + 2e^{x})) - x(x + 2)(c_1 + c_2(xe^{x} + e^{x})) + (x + 2) (c_1x + c_2 xe^{x}) = \\
	c_2 (x^{3}e^{x} + 2x^{2}e^{x}) - \\
	[c_1 x^{2} + c_2 (x^{3} e^{x}+ x^{2} e^{x}) + 2 c_1 x + 2c_2 (x^{2} e^{x} + xe^{x})]
	+ c_1 x^{2} + c_2 x^{2} e^{x} + 2c_1x + c_2 xe^{x} = 0
\end{align*}
as solu\c c\~oes est\~ao verificadas. Calculando o wronskiano temos que
\[ W = \left  | 
	\begin{array}{cc}
		x & 1 \\
		xe^{x} & xe^{x} + e^{x}
	\end{array}
	\right |  = x^{2}e^{x} \neq 0 \Rightarrow \mbox{L.I}
\]
a solu\c c\~ao particular \'e da forma 
\begin{align*}
	y_p(x) = u_1 y_1 + u_2 y_2 
\end{align*}
vamos calcular $u_1$ e $u_2$
\begin{align*}
	u_1 &= - \int \frac{y_2 f}{a_2 W} dx = - \int 2dx = -2x + k_1 \\
	u_2 &=  \int \frac{y_1 f}{a_2 W} dx = \int 2e^{-x} dx = -2e^{-x} + k_2 
\end{align*}

c)
\begin{align*}
	xy^{(2)} - (1 + x)y^{(1)} + y = x^{2}e^{x}, x > 0 \\
	y_1 = 1 + x, y_2 = e^{x}
\end{align*}
vamos verificar que $y_1$ e $y_2$ satisfazem a equa\c c\~ao homog\^enea
\begin{align*}
	y_h = c_1 (1 + x) + c_2 e^{x} \\
	{y_h}^{(1)} = c_1 + c_2e^{x} \\
	{y_h}^{(2)} = c_2e^{x} \\
\end{align*}
substituindo temos que 
\begin{align*}
	x(c_2 e^{x}) - (1+x)(c_1 + c_2 e^{x}) + c_1 (1 + x) + c_2 e^{x} =\\
	= c_2 x e^{x} -\left [ c_1 + c_2 e^{x} + c_1 x + c_2 xe^{x}\right ] + c_1 + xc_1 + c_2 e^{x}  = 0
\end{align*}
a solu\c c\~ao foi verificada. Calculando o wronskiano 
\[ W = \left [
	\begin{array}{cc}
		1 + x & 1 \\
		e^{x} & e^{x}
	\end{array}
\right ]  = xe^{x} \neq 0 \Rightarrow \mbox{L.I}
\]
a solu\c c\~ao particular \'e da forma 
\begin{align*}
	y_p = u_1 y_1 + u_2 y_2
\end{align*}
vamos encontra $u_1$ e $u_2$ segue que 
\begin{align*}
	u_1 &= -\int \frac{y_2 f}{a_2 W} dx = -\int e^{x} dx = -e^{x} + k_1 \\
	u_2 &=  \int \frac{y_1 f}{a_2 W} dx  = \int (1 + x) dx = x + \frac{x^{2}}{2} + k_2
\end{align*}

d)
\begin{align*}
	(1 - x)y^{(2)} + xy^{(1)} - yx  = 2{(x -1)}^{2}e^{-x}, x > 1 \\
	y_1 = e^{x}, y_2 = x
\end{align*}
vamos verificar que $y_1$ e $y_2$ s\~ao solu\c c\~oes da homog\^enea, segue que 
\begin{align*}
	y_h &= c_1 e^{x} + c_2 x \\
	{y_h}^{(1)} &= c_1 e^{x} + c_2  \\
	{y_h}^{2} &= c_1 e^{x}  \\
\end{align*}
vamos substituir 
\begin{align*}
	(1 - x)(c_1 e^{x}) + x(c_1 e^{x} + c_2) - (c_1 e^{x} + c_2 x) = 0
\end{align*}
solu\c c\~ao verificada. Calculando o wronskiano
\[ W = 
\left | 
\begin{array}{cc}
	e^{x} & e^{x} \\
	x & 1
\end{array} 
\right | = e^{x}(1 - x) \neq 0 \Rightarrow \mbox{L.I}
\]
a solu\c c\~ao particular \'e da forma 
\begin{align*}
	y_p = u_1 y_1 + u_2 y_2 
\end{align*}
vamos encontra $u_1$ e $u_2$
\begin{align*}
	u_1 &= -\int \frac{y_2 f}{a_2 W} dx = -2 \int xe^{-2x} dx =  \frac{1}{2} e^{-2x} (2x + 1) + k_1 \\
	u_2 &= \int \frac{y_1 f}{a_2 W} dx = \int 2 e^{-x} dx = -2 e^{-x} + k_2
\end{align*}

e) 
\begin{align*}
	x^{2}y^{(2)} -3xy^{1} + 4y = x^{2}ln(x),  x > 0 \\
	y_1 = x^{2}, y_2 = x^{2} ln(x)
\end{align*}
vamos verificar que \'e solu\c c\~ao, segue que 
\begin{align*}
	y_h &= c_1 x^{2} + c_2 x^{2} ln(x) \\
	{y_h}^{(1)} &= 2c_1 x + c_2 (x + 2x ln(x)) \\
	{y_h}^{(2)} &= 2c_1 + c_2 (3 + 2ln(x)) 
\end{align*}
substituindo temos que 
\begin{align*}
	x^{2} [ 2 c_1 + c_2 (3 + 2ln(x))] -3x[2c_1 x + c_2 (x + 2ln(x))] + 4 [ c_1 x^{2} + c_2 x^{2} ln(x)] = \\
	= 2x^{2}c_1 + 3x^{2}c_2 + 2x^{2}c_2 ln(x) -6 c_1 x^{2} -3c_2 x^{2} -6x^{2}ln(x) + 4c_1 x^{2} + 4c_2 x^{2} ln(x)
	= 0
\end{align*}
solu\c c\~ao verificada. Vamos calcular o wronskiano
\[\left |
	\begin{array}{cc}	
		x^{2} & 2x \\
		x^{2}ln(x) & x + 2xln(x)
	\end{array}
	\right | = x^{3} \neq 0 \Rightarrow \mbox{L.I}
\]
a solu\c c\~ao particular \'e da forma 
\begin{align*}
	y_p = u_1 y_1 + u_2 y_2
\end{align*}
vamos calcular $u_1$ e $u_2$
\begin{align*}
	u_1 &= - \int \frac{y_2 f}{a_2 W} dx = - \int \frac{ln^{2} (x)}{x} dx = -\frac{1}{3} ln^{3} (x) + k_1 \\
	u_2 &= \int \frac{y_1 f}{a_2 W} dx = \int \frac{ln (x)}{x} dx  = \frac{1}{2} ln^{2} (x) + k_2
\end{align*}

f)
\begin{align*}
	x^{2}y^{(2)} + xy^{(1)} + (x^{2} - 0,25)y = 3x^{\frac{3}{2}}sen(x), x > 0 \\
	y_1 = x^{\frac{-1}{2}} sen(x), y_2 = x^{\frac{-1}{2}} cos(x)
\end{align*}
vamos verificar que \'e solu\c c\~ao, temos que 
\begin{align*}
	y_h &= c_1 x^{\frac{-1}{2}} sen(x) + c_2  x^{\frac{-1}{2}} cos(x)\\
	{y_h}^{(1)} &= c_1 [x^{\frac{-1}{2}} cos(x) -\frac{1}{2} x^{\frac{-3}{2}} sen(x)] + c_2
	[-x^{\frac{-1}{2}} sen(x) -\frac{1}{2}x^{\frac{-3}{2}} cos(x) ] \\
	{y_h}^{(2)} &= c_1[ -x^{\frac{-1}{2}} sen(x) -\frac{1}{2} x^{\frac{-3}{2}} cos(x) - \frac{1}{2}
	x^{\frac{-3}{2}} cos(x) + \frac{3}{4} x^{\frac{-5}{2}} sen(x)] \\
	&+ c_2 [-x^{\frac{-1}{2}}cos(x) + \frac{1}{2}x^{\frac{-3}{2}} sen(x) + \frac{1}{2} x^{\frac{-3}{2}} sen(x) +
	\frac{3}{4} x^{\frac{-5}{2}} cos(x)]
\end{align*}
substituindo temos que 
\begin{align*}
	&c_1[ -x^{\frac{3}{2}} sen(x) -x^{\frac{1}{2}} cos(x) + \frac{3}{4} x^{\frac{-1}{2}} sen(x)] \\
	+& c_2[ -x^{\frac{3}{2}}cos(x) + x^{\frac{1}{2}} sen(x) + \frac{3}{4} x^{\frac{-1}{2}} cos(x)] \\
	+& c_1 [ x^{\frac{1}{2}} cos(x) - \frac{1}{2}x^{\frac{-1}{2}}sen(x)] \\
	+& c_2[ -x^{\frac{1}{2}} sen(x) -\frac{1}{2}x^{\frac{-1}{2}}cos(x)] \\
	+&c_1 [ x^{\frac{3}{2}} sen(x)] + c_2 [ x^{\frac{3}{2}}cos(x)] \\
	-&\frac{1}{4} [c_1(x^{\frac{-1}{2}}sen(x) + c_2 x^{\frac{-1}{2}}cos(x))] = 0
\end{align*}
de fato \'e solu\c c\~ao. Vamos ao c\'alculo do wronskiano

\[ W = \left |
	\begin{array}{cc}	
		x^{\frac{-1}{2}}sen(x)  &  x^{ \frac{-1}{2}} cos(x) \\
		-\frac{1}{2}x^{\frac{-3}{2}} sen(x) + x^{\frac{-1}{2}}cos(x) & -\frac{1}{2}x^{\frac{-3}{2}} cos(x)
		-x^{\frac{-1}{2}}sen(x) 	
	\end{array}
	\right | =\frac{-1}{x} \neq 0, \forall x > 0
\]
a solu\c c\~ao particular tem a forma 
\begin{align*}
	y_p = u_1 y_1 + u_2 y_2 
\end{align*}
vamos encontrar $u_1$ e $u_2$
\begin{align*}
	u_1 &= -\int \frac{y_2 f}{a_2 W}dx = \int 3 sen(x) cos(x) dx = \frac{3sen^{2}x}{2} + k1 \\
	u_2 &= \int \frac{y_1 f}{a_2 W}dx = -\frac{3}{2}\left [ x - \frac{sen(x)}{2}  + k_2 \right ] 
\end{align*}

\textbf{7. Quest\~ao:}\\
Primeiramente vamos verificar que $y_1 = sin(\frac{1}{2})$ e $y_2 = cos(\frac{1}{2})$ s\~ao solu\c c\~oes, portanto a
solu\c c\~ao homog\^enea \'e da forma 
\begin{align*}
	y_h &= c_1 sen(\frac{1}{x}) + c_2 cos(\frac{1}{2}) \\
	{y_h}^{(1)} &= \frac{1}{x^{2}} (-c_1 cos (\frac{1}{x}) + c_2 sen (\frac{1}{x})) \\
	{y_h}^{(2)} &= \frac{1}{x^{4}}[ -c_1 sen(\frac{1}{x}) + 2c_1xcos(\frac{1}{x}) -c_2 cos(\frac{1}{x}) -2c_2
	xsen(\frac{1}{x})]
\end{align*}
substituindo temos que 
\begin{align*}
	&-c_1 sen(\frac{1}{x}) + c_1 x cos(\frac{1}{x}) -c_2cos(\frac{1}{x}) -2c_2xsen(\frac{1}{x}) \\
	& -2c_1 xcos(\frac{1}{x}) +2c_2 x sen(\frac{1}{x}) \\
	& + c_1 sen(\frac{1}{x}) + c_2 cos(\frac{1}{x}) = 0
\end{align*}
de fato \'e solu\c c\~ao. Calculando o wronskiano para verificar se s\~ao L.I.
\[ W = \left |
	\begin{array}{cc}	
		sen(\frac{1}{x}) & -\frac{1}{x^{2}} cos(\frac{1}{x}) \\
		cos(\frac{1}{x}) & \frac{1}{x^{2}} sen(\frac{1}{x})
	\end{array}
	\right |  = \frac{1}{x^{2}} \neq 0, x\neq 0 \Rightarrow \mbox{L.I}
\]
vamos encontrar $y(x)$ tal que $y(\frac{1}{x}) = 1$ e $y^{(1)}(\frac{1}{x}) = -1.$ 
Usando $y$ e sua derivada obtemos o seguite sistema
\[\left \{ 
	\begin{array}{cc}	
		c_1 sen(\pi) + c_2 cos(\pi) = 1  \Rightarrow c_2 = -1 \\\
		-c_1cos(\pi) + c_2 sen(\pi) = -\frac{1}{{\pi}^{2}} \Rightarrow c_1 = -\frac{1}{{\pi}^{2}}
	\end{array}
	\right . 
\]
portanto, a  $y$ procurada \'e da forma 
\begin{align*}
	y = -\frac{1}{{\pi}^{2}} sen(\frac{1}{x}) -cos(\frac{1}{x})
\end{align*}

\textbf{8. Quest\~ao:}\\
Seja f,g e h as fun\c c\~oes diferenci\'aveis em $\mathbb{R}$, devemos provar que 
\begin{align*}
	W(fg,fh) = f^{2}W(g,h)
\end{align*}
de fato aplicando o wronskiano temos que
\[ W (fg,fh) = \left  |
	\begin{array}{cc}
		fg & fg' + gf' \\
		fh & fh' + hf'
	\end{array}
	\right  | = fg(fh' + hf')  -fh (fg' + gf') = f^{2} gh' - f^{2} g'h = f^{2}(gh' -g'h) 
\]
por outro lado temos que 
\[ f^{2} W(g,h) = f^{2}\left |
	\begin{array}{cc}
		g & g' \\
		h & h'
	\end{array}
	\right |  = f^{2} (gh' - g'h)
\]
o que completa a prova.

\textbf{9. Quest\~ao:}\\
Temos que 
\begin{align*}
	y_3 &= a_1 y_1 + a_2 y_2 \\
	y_4 &= b_1 y_1 + b_2 y_2
\end{align*}
aplicando o wronskino temos 
\[W = \left | 
	\begin{array}{cc}
		y_3 & y_4 \\
		{y_3}^{'} & {y_4}^{'}
	\end{array}
	\right | \neq 0 \Rightarrow (a_1 y_1 + a_2 y_2)( b_1 {y_1}^{'} + b_2 {y_2}^{'}) - (b_1 y_1 + b_2 y_2)( a_1
	{y_1}^{'} + a_2 {y_2}^{'}) \neq 0
\]
arrumando temos que as seguinte rela\c c\~es 
\begin{align*}
	y_2 {y_1}^{'} \neq 0 \\
	a_2 b_1 \neq b_2 a_1
\end{align*}
\end{document}

